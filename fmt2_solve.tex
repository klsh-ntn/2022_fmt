% arara: xelatex
\documentclass[12pt]{article} % размер шрифта

\usepackage{tikz} % картинки в tikz
\usepackage{microtype} % свешивание пунктуации

\usepackage{array} % для столбцов фиксированной ширины

\usepackage{indentfirst} % отступ в первом параграфе

\usepackage{multicol} % текст в несколько колонок
\usepackage{verbatim}

\graphicspath{{images/}} % путь к картинкам

\usepackage{sectsty} % для центрирования названий частей
\allsectionsfont{\centering} % приказываем центрировать все sections

\usepackage{amsmath, amssymb} % куча стандартных математических плюшек

\usepackage[top=2cm, left=1.5cm, right=1.5cm, bottom=2cm]{geometry} % размер текста на странице

\usepackage{lastpage} % чтобы узнать номер последней страницы

\usepackage{enumitem} % дополнительные плюшки для списков
%  например \begin{enumerate}[resume] позволяет продолжить нумерацию в новом списке
\usepackage{caption} % подписи к картинкам без плавающего окружения figure


\usepackage{fancyhdr} % весёлые колонтитулы
\pagestyle{empty}
%\lhead{ФМТ}
%\chead{}
%\rhead{КЛШ-2019}
%\lfoot{}
%\cfoot{}
%\rfoot{\thepage/\pageref{LastPage}}
%\renewcommand{\headrulewidth}{0.4pt}
%\renewcommand{\footrulewidth}{0.4pt}



\usepackage{todonotes} % для вставки в документ заметок о том, что осталось сделать
% \todo{Здесь надо коэффициенты исправить}
% \missingfigure{Здесь будет картина Последний день Помпеи}
% команда \listoftodos — печатает все поставленные \todo'шки

\usepackage{booktabs} % красивые таблицы
% заповеди из документации:
% 1. Не используйте вертикальные линии
% 2. Не используйте двойные линии
% 3. Единицы измерения помещайте в шапку таблицы
% 4. Не сокращайте .1 вместо 0.1
% 5. Повторяющееся значение повторяйте, а не говорите "то же"

\usepackage{fontspec} % поддержка разных шрифтов
\usepackage{polyglossia} % поддержка разных языков

\setmainlanguage{russian}
\setotherlanguages{english}

\setmainfont{Linux Libertine O} % выбираем шрифт
% можно также попробовать Helvetica, Arial, Cambria и т.Д.

% чтобы использовать шрифт Linux Libertine на личном компе,
% его надо предварительно скачать по ссылке
% http://www.linuxlibertine.org/

\newfontfamily{\cyrillicfonttt}{Linux Libertine O}
% пояснение зачем нужно шаманство с \newfontfamily
% http://tex.stackexchange.com/questions/91507/

\AddEnumerateCounter{\asbuk}{\russian@alph}{щ} % для списков с русскими буквами
\setlist[enumerate, 2]{label=\asbuk*),ref=\asbuk*} % списки уровня 2 будут буквами а) б) ...

%% эконометрические и вероятностные сокращения
\DeclareMathOperator{\Cov}{Cov}
\DeclareMathOperator{\Corr}{Corr}
\DeclareMathOperator{\Var}{Var}
\DeclareMathOperator{\E}{\mathbb{E}}
\newcommand \hVar{\widehat{\Var}}
\let\P\relax
\DeclareMathOperator{\P}{\mathbb{P}}
\let\Re\relax
\DeclareMathOperator{\Re}{Re}



\begin{document}
\begin{center}
\begin{tabular}{cc}
\includegraphics[scale=0.25]{klshlogo.pdf} &
\raisebox{1cm}{
  {\Large\bf ФМТ, II тур}
}
\end{tabular}
\end{center}

\begin{enumerate}
\item %  На пружинных весах стоит сосуд с водой; весы показывают 1 кг. На невесомом подвесе, соединённом с динамометром, висит железная гирька массой 100\,г ($\rho_{\textsf{Fe}}=8\,\mfrac{\textrm{г}}{\textrm{см}^3}$). 
% Гирьку медленно опускают в сосуд с водой до полного погружения; вода при этом не вытекает через край. Что покажут весы? %покажет динамометр?
На двух чашках чашечных весов стоят сосуды содержащие по одному литру воды. 
В одном сосуде плавает пробковый шарик плотности $700 \text{кг/м}^3$.
На дне второго сосуда лежит металлический кубик плотностью $5600 \text{кг/м}^3$.
Объем кубика в 8 раз меньше объема шарика. 

Какую разницу весов показывают чашечные весы?

\textsl{Ответ/решение}: 
Весы покажут нулевую разницу. Засчитываем ответы «0 кг», «0 Н» и «отсутствие разницы».
Аргумент: масса шарика и кубика одинаковая. Положим их мысленно в пустые сосуды и затем дольем по литру воды.
% Весы покажут сумму весов воды и гирьки, т.\,е. 1100 грамм. Показания весов не зависят от того, находится гирька на дне, либо не касается такового, либо вообще стоит на весах рядом с сосудом. \bigskip

\item Три последовательных натуральных 2022-значных числа записаны подряд (встык) так, что получилось 6066-значное число~$N$. 
На какое минимальное простое всегда делится~$N$?

\textsl{Ответ/решение}: Остатки от деления любых трёх последовательных натуральных чисел на 3 равны (с точностью до перестановки) 0, 1 и 2. 
Следовательно, конкатенация трёх последовательных натуральных всегда делится на три.\bigskip

Доказательства того, что сумма цифр и само число дают одинаковый остаток, не требуем. 

%\noindent\begin{minipage}{18.3cm}\begin{wrapfigure}[7]{r}{75mm}\raisebox{-16pt}[62pt][38pt]{\includegraphics[width=74mm]{ugol_gruz}}\end{wrapfigure}
\item Автобусы из Красноярска в \textit{Орбиту} первую половину пути прошли во скоростью~$v \text{км/час}$, а вторую~--- с вдвое большей скоростью. 
С какой средней скоростью автобусы шли из Красноярска в \textit{Орбиту}?
 %Груз массы $M$ подвешен как показано на рисунке и находится в покое; угол между нитями составляет $90^\circ$, а один из углов между нитью и потолком в два раза меньше другого. Массы нитей пренебрежимо малы по сравнению с массой груза. Найти силу натяжения нитей.
%\end{minipage}\vspace{\itemsep}

\textsl{Ответ/решение}: Средняя скорость — это НЕ среднее арифметичекое! 
Это среднее гармоническое: 
\[
  \langle v \rangle = \frac{\ell}{\frac{\ell}{2v}+\frac{\ell}{4v}} = \frac{4v}{3}.
\]

%Когда шар опускался, то на него действовала сила тяжести $(M+m)g- F_{\textrm{Архимеда}}-F_{\textrm{трения}}$. Когда груз выбросили, шар стал подниматься и на него стали действовать силы $F_{\textrm{Архимеда}} - F_{\textrm{трения}}-Mg$. Дальнейшее понятно.\bigskip


\item  Дан квадрат $ABCD$; прямая, пересекающая две смежных стороны квадрата, делит его на две части, площади которых относятся как $1\div 2022$.
Прямая делит одну из пересекаемых сторон в пропорции $1\div 2$. 

В каком соотношении прямая делит другую пересекаемую сторону?

\textsl{Ответ/решение}: Обозначим кусочки одной стороны как $x$ и $2x$, другой стороны — как $y$ и $ky$. 
Получаем два варианта системы
\[
\begin{cases}
3x(1+k)y=2023S \\
0.5xy=S
\end{cases}
\]  
Отсюда $k=2017/6$.
Либо
\[
\begin{cases}
3x(1+k)y=2023S \\
xy=S
\end{cases}
\]  
Отсюда $k=2020/3$.

Важно: если заявлено одно решение, то это 1 балл за решение и переход. 
Например, возможен сценарий: команда А демонстрирует верно одно решение, команда Б делает замечание, что второе решение не дано.
Подача переходит. Команда Б заявляет второе решение с явной отсылкой на то, что первое уже было изложено. Счет будет 3:1 в пользу Б. 

\end{enumerate}


\end{document}
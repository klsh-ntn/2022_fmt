% arara: xelatex
\documentclass[12pt]{article} % размер шрифта

\usepackage{tikz} % картинки в tikz
\usepackage{microtype} % свешивание пунктуации

\usepackage{array} % для столбцов фиксированной ширины

\usepackage{indentfirst} % отступ в первом параграфе

\usepackage{multicol} % текст в несколько колонок
\usepackage{verbatim}

\graphicspath{{images/}} % путь к картинкам

\usepackage{sectsty} % для центрирования названий частей
\allsectionsfont{\centering} % приказываем центрировать все sections

\usepackage{amsmath, amssymb} % куча стандартных математических плюшек

\usepackage[top=2cm, left=1.5cm, right=1.5cm, bottom=2cm]{geometry} % размер текста на странице

\usepackage{lastpage} % чтобы узнать номер последней страницы

\usepackage{enumitem} % дополнительные плюшки для списков
%  например \begin{enumerate}[resume] позволяет продолжить нумерацию в новом списке
\usepackage{caption} % подписи к картинкам без плавающего окружения figure


\usepackage{fancyhdr} % весёлые колонтитулы
\pagestyle{empty}
%\lhead{ФМТ}
%\chead{}
%\rhead{КЛШ-2019}
%\lfoot{}
%\cfoot{}
%\rfoot{\thepage/\pageref{LastPage}}
%\renewcommand{\headrulewidth}{0.4pt}
%\renewcommand{\footrulewidth}{0.4pt}



\usepackage{todonotes} % для вставки в документ заметок о том, что осталось сделать
% \todo{Здесь надо коэффициенты исправить}
% \missingfigure{Здесь будет картина Последний день Помпеи}
% команда \listoftodos — печатает все поставленные \todo'шки

\usepackage{booktabs} % красивые таблицы
% заповеди из документации:
% 1. Не используйте вертикальные линии
% 2. Не используйте двойные линии
% 3. Единицы измерения помещайте в шапку таблицы
% 4. Не сокращайте .1 вместо 0.1
% 5. Повторяющееся значение повторяйте, а не говорите "то же"

\usepackage{fontspec} % поддержка разных шрифтов
\usepackage{polyglossia} % поддержка разных языков

\setmainlanguage{russian}
\setotherlanguages{english}

\setmainfont{Linux Libertine O} % выбираем шрифт
% можно также попробовать Helvetica, Arial, Cambria и т.Д.

% чтобы использовать шрифт Linux Libertine на личном компе,
% его надо предварительно скачать по ссылке
% http://www.linuxlibertine.org/

\newfontfamily{\cyrillicfonttt}{Linux Libertine O}
% пояснение зачем нужно шаманство с \newfontfamily
% http://tex.stackexchange.com/questions/91507/

\AddEnumerateCounter{\asbuk}{\russian@alph}{щ} % для списков с русскими буквами
\setlist[enumerate, 2]{label=\asbuk*),ref=\asbuk*} % списки уровня 2 будут буквами а) б) ...

%% эконометрические и вероятностные сокращения
\DeclareMathOperator{\Cov}{Cov}
\DeclareMathOperator{\Corr}{Corr}
\DeclareMathOperator{\Var}{Var}
\DeclareMathOperator{\E}{\mathbb{E}}
\newcommand \hVar{\widehat{\Var}}
\let\P\relax
\DeclareMathOperator{\P}{\mathbb{P}}
\let\Re\relax
\DeclareMathOperator{\Re}{Re}



\begin{document}
\begin{center}
\begin{tabular}{cc}
\includegraphics[scale=0.25]{klshlogo.pdf} &
\raisebox{1cm}{
  {\Large\bf ФМТ, II тур}
}
\end{tabular}
\end{center}

\begin{enumerate}
\item На двух чашках чашечных весов стоят сосуды содержащие по одному литру воды. 
В одном сосуде плавает пробковый шарик плотности $700\text{ кг/м}^3$.
На дне второго сосуда лежит металлический кубик плотностью $5600 \text{ кг/м}^3$.
Объем кубика в 8 раз меньше объема шарика. 

Какую разницу весов показывают чашечные весы?

\item Три последовательных натуральных 2022-значных числа записаны подряд (встык) так, что получилось 6066-значное число~$N$. 
На какое минимальное простое число всегда делится~$N$?

\item Автобусы из Красноярска в \textit{Орбиту} первую половину пути прошли во скоростью~$v \text{ км/час}$, а вторую — с вдвое большей скоростью. 
С какой средней скоростью автобусы шли из Красноярска в \textit{Орбиту}?


\item  Дан квадрат $ABCD$; прямая, пересекающая две смежных стороны квадрата, делит его на две части, площади которых относятся как $1\div 2022$.
Прямая делит одну из пересекаемых сторон в пропорции $1\div 2$. 

В каком соотношении прямая делит другую пересекаемую сторону?

\end{enumerate}

\vfill

\begin{center}
  \begin{tabular}{cc}
  \includegraphics[scale=0.25]{klshlogo.pdf} &
  \raisebox{1cm}{
    {\Large\bf ФМТ, II тур}
  }
  \end{tabular}
  \end{center}
  

\begin{enumerate}
  \item На двух чашках чашечных весов стоят сосуды содержащие по одному литру воды. 
  В одном сосуде плавает пробковый шарик плотности $700 \text{ кг/м}^3$.
  На дне второго сосуда лежит металлический кубик плотностью $5600 \text{ кг/м}^3$.
  Объем кубика в 8 раз меньше объема шарика. 
  
  Какую разницу весов показывают чашечные весы?
  
  \item Три последовательных натуральных 2022-значных числа записаны подряд (встык) так, что получилось 6066-значное число~$N$. 
  На какое минимальное простое число всегда делится~$N$?
  
  \item Автобусы из Красноярска в \textit{Орбиту} первую половину пути прошли во скоростью~$v \text{ км/час}$, а вторую — с вдвое большей скоростью. 
  С какой средней скоростью автобусы шли из Красноярска в \textit{Орбиту}?
  
  
  \item  Дан квадрат $ABCD$; прямая, пересекающая две смежных стороны квадрата, делит его на две части, площади которых относятся как $1\div 2022$.
  Прямая делит одну из пересекаемых сторон в пропорции $1\div 2$. 
  
  В каком соотношении прямая делит другую пересекаемую сторону?
  
  \end{enumerate}
  



\end{document}